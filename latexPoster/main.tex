%%%%%%%%%%%%%%%%%%%%%%%%%%%%%%%%%%%%%%%%%
% Jacobs Landscape Poster
% LaTeX Template
% Version 1.0 (29/03/13)
%
% Created by:
% Computational Physics and Biophysics Group, Jacobs University
% https://teamwork.jacobs-university.de:8443/confluence/display/CoPandBiG/LaTeX+Poster
% 
% Further modified by:
% Nathaniel Johnston (nathaniel@njohnston.ca)
%
% This template has been downloaded from:
% http://www.LaTeXTemplates.com
%
% License:
% CC BY-NC-SA 3.0 (http://creativecommons.org/licenses/by-nc-sa/3.0/)
%
%%%%%%%%%%%%%%%%%%%%%%%%%%%%%%%%%%%%%%%%%

%----------------------------------------------------------------------------------------
%	PACKAGES AND OTHER DOCUMENT CONFIGURATIONS
%----------------------------------------------------------------------------------------

\documentclass[final]{beamer}

\usepackage[scale=1.24]{beamerposter} % Use the beamerposter package for laying out the poster



\usepackage{url}

 \renewcommand{\UrlFont}{\small\tt}



\usetheme{confposter} % Use the confposter theme supplied with this template

\setbeamercolor{block title}{fg=ngreen,bg=white} % Colors of the block titles
\setbeamercolor{block body}{fg=black,bg=white} % Colors of the body of blocks
\setbeamercolor{block alerted title}{fg=white,bg=dblue!70} % Colors of the highlighted block titles
\setbeamercolor{block alerted body}{fg=black,bg=dblue!10} % Colors of the body of highlighted blocks
% Many more colors are available for use in beamerthemeconfposter.sty

%-----------------------------------------------------------
% Define the column widths and overall poster size
% To set effective sepwid, onecolwid and twocolwid values, first choose how many columns you want and how much separation you want between columns
% In this template, the separation width chosen is 0.024 of the paper width and a 4-column layout
% onecolwid should therefore be (1-(# of columns+1)*sepwid)/# of columns e.g. (1-(4+1)*0.024)/4 = 0.22
% Set twocolwid to be (2*onecolwid)+sepwid = 0.464
% Set threecolwid to be (3*onecolwid)+2*sepwid = 0.708

\newlength{\sepwid}
\newlength{\onecolwid}
\newlength{\twocolwid}
\newlength{\threecolwid}
\setlength{\paperwidth}{48in} % A0 width: 46.8in
\setlength{\paperheight}{36in} % A0 height: 33.1in
\setlength{\sepwid}{0.024\paperwidth} % Separation width (white space) between columns
\setlength{\onecolwid}{0.22\paperwidth} % Width of one column
\setlength{\twocolwid}{0.464\paperwidth} % Width of two columns
\setlength{\threecolwid}{0.708\paperwidth} % Width of three columns
\setlength{\topmargin}{-0.5in} % Reduce the top margin size
%-----------------------------------------------------------

\usepackage{graphicx}  % Required for including images

\usepackage{booktabs} % Top and bottom rules for tables

%----------------------------------------------------------------------------------------
%	TITLE SECTION 
%----------------------------------------------------------------------------------------

\title{My Stuff} % Poster title

\author{Authors} % Author(s)

\institute{Wentworth Institute of Technology, Department of Applied Mathematics} % Institution(s)

%----------------------------------------------------------------------------------------

\begin{document}

\addtobeamertemplate{block end}{}{\vspace*{2ex}} % White space under blocks
\addtobeamertemplate{block alerted end}{}{\vspace*{2ex}} % White space under highlighted (alert) blocks

\setlength{\belowcaptionskip}{2ex} % White space under figures
\setlength\belowdisplayshortskip{2ex} % White space under equations

\begin{frame}[t] % The whole poster is enclosed in one beamer frame

\begin{columns}[t] % The whole poster consists of three major columns, the second of which is split into two columns twice - the [t] option aligns each column's content to the top

\begin{column}{\sepwid}\end{column} % Empty spacer column

\begin{column}{\onecolwid} % The first column

%----------------------------------------------------------------------------------------
%	OBJECTIVES
%----------------------------------------------------------------------------------------

\begin{alertblock}{Objectives}
{\small
Blah Blah
\begin{itemize}
\item item 1
\item item 2
\item item 3
\end{itemize}
}
\end{alertblock}

%----------------------------------------------------------------------------------------
%	INTRODUCTION
%----------------------------------------------------------------------------------------

\begin{block}{Abstract}
{\small
\vskip20em
}
\end{block}
\begin{alertblock}{Important Result}
{\small
Super great!!!
}
\end{alertblock} 
%------------------------------------------------

%\begin{figure}
%\includegraphics[width=0.65\linewidth]{dist1.png}
%\caption{Super Kewl}
%\end{figure}

%----------------------------------------------------------------------------------------

\end{column} % End of the first column

\begin{column}{\sepwid}\end{column} % Empty spacer column

\begin{column}{\twocolwid} % Begin a column which is two columns wide (column 2)

\begin{columns}[t,totalwidth=\twocolwid] % Split up the two columns wide column

\begin{column}{\onecolwid}\vspace{-.6in} % The first column within column 2 (column 2.1)

%\begin{figure}
%\includegraphics[width=0.6\linewidth]{dist2.png}
%\caption{Above we see $\cdots$}
%\end{figure}


%----------------------------------------------------------------------------------------

\end{column} % End of column 2.1

\begin{column}{\onecolwid}\vspace{-.6in} % The second column within column 2 (column 2.2)

%----------------------------------------------------------------------------------------
%	METHODS
%----------------------------------------------------------------------------------------

\begin{block}{Methods}
{\small
\begin{eqnarray*}
&& u_{hsbt} = v_{hsbt} + \alpha_{h}\cdot e(b,p_{hst}) + \gamma_{h}(d_{hs}) + X_{hs}\cdot \beta +\xi_{hs}+\epsilon_{hst} \\
&& \mbox{Let $t$ and $b$ be constant}\\
&& \mbox{Then $v_{hsbt}$ is constant}\\
&& \mbox{Now considering $\alpha_{h}\cdot e(b,p_{hst})$:}\\
&&      \mbox{let } I_{h} = \mbox{income of household $h$}\\
&&      \Delta \frac{p_{hs}}{I_{h}} \propto \Delta u_{hs}\\
&&      \frac{\Delta u_{hs}}{\Delta p_{hs}} \propto \frac{1}{I_{h}}\\
&&      \frac{\delta u_{hs}}{\delta p_{hs}} = \frac{c}{I_{h}}\\
&&      \alpha_h \cdot e=u_{hs} = \frac{c\cdot p_{hs}}{I_{h}}\\
&&      \mbox{Consider $c=-1$ since the utility should decrease as price increases}\\
&& 	    \alpha_h \cdot e = \frac{ -p_{hst}}{I_{h}}\\
\end{eqnarray*}

\begin{itemize}
\item 
\item
\item \end{itemize}
}
\end{block}

%----------------------------------------------------------------------------------------

%\end{column} % End of column 2.2

%\end{columns} % End of the split of column 2 - any content after this will now take up 2 columns width

%----------------------------------------------------------------------------------------
%	IMPORTANT RESULT
%----------------------------------------------------------------------------------------


\end{column} % End of column 2.2

\end{columns} % End of the split of column 2 - any content after this will now take up 2 columns width
%----------------------------------------------------------------------------------------

\begin{columns}[t,totalwidth=\twocolwid] % Split up the two columns wide column again

\begin{column}{\onecolwid} % The first column within column 2 (column 2.1)

%----------------------------------------------------------------------------------------
%	MATHEMATICAL SECTION
%----------------------------------------------------------------------------------------

\begin{block}{Background}
{\small
%\cite{Frank,Joseph}:
}
\end{block}

%----------------------------------------------------------------------------------------

\end{column} % End of column 2.1

\begin{column}{\onecolwid} % The second column within column 2 (column 2.2)

%----------------------------------------------------------------------------------------
%	RESULTS
%----------------------------------------------------------------------------------------

\begin{block}{Results}

%\begin{figure}
%\includegraphics[width=0.8\linewidth]{circle33.png}
%\caption{Great}
%\end{figure}
%{\small
%\begin{table}[h]
%\begin{tabular}{|l|l|l|l|}
%\hline
%col1 & col2& col3      & col4    \\ \hline
%a      & 1   & 2& 3  \\ \hline
%b & 44   & 4 & 7\\ \hline
%\end{tabular}
%\end{table}



\end{block}

%----------------------------------------------------------------------------------------

\end{column} % End of column 2.2

\end{columns} % End of the split of column 2

\end{column} % End of the second column

\begin{column}{\sepwid}\end{column} % Empty spacer column

\begin{column}{\onecolwid} % The third column

%----------------------------------------------------------------------------------------
%	CONCLUSION
%----------------------------------------------------------------------------------------

\begin{block}{Conclusion}
{\small
\begin{itemize}
\item 
\item f
\item a
\end{itemize}

}
\end{block}

%----------------------------------------------------------------------------------------
%	FUTURE WORK
%----------------------------------------------------------------------------------------



\begin{block}{Future Work}
{\small
 
}
\end{block}

%----------------------------------------------------------------------------------------
%	REFERENCES
%----------------------------------------------------------------------------------------

\begin{block}{References}
{\small
%\nocite{*} % Insert publications even if they are not cited in the poster- avoid

%\small{\bibliographystyle{unsrt}
%\bibliography{sample}\vspace{0.5in}}
}
\end{block}



%----------------------------------------------------------------------------------------
%	CONTACT INFORMATION
%----------------------------------------------------------------------------------------

\setbeamercolor{block alerted title}{fg=black,bg=norange} % Change the alert block title colors
\setbeamercolor{block alerted body}{fg=black,bg=white} % Change the alert block body colors

\begin{alertblock}{Contact Information}

\begin{itemize}
\item Email: \href{mailto:YES@wit.edu}{yes@wit.edu}
\item Email: \href{mailto:YES@wit.edu}{yes@.edu}
\item
\end{itemize}

\end{alertblock}



%----------------------------------------------------------------------------------------

\end{column} % End of the third column

\end{columns} % End of all the columns in the poster

\end{frame} % End of the enclosing frame

\end{document}